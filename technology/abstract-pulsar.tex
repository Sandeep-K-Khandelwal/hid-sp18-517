\section{Pulsar}

Apache Pulsar which is also an open source project of the Apache
foundation was originally developed by Yahoo. It is a messaging
solution that enables high performance server to server messaging.
Similar to Kafka Pulsar is based on publisher-subscriber model.  Some
of the key features of Pulsar include low latency in publishing,
guaranteed message delivery, scalability and so on.  The
publish-subscribe pattern involves components such as producers,
consumers, topics and subscription wherein; topics are channels that
transmit data from source to target or in other words from producers
to consumers, producers job is to publish a message and a consumer
process is the one that receives the message.  Subscriptions are set
of rules that determine how messages flow in the system from producers
to consumers and have three modes namely exclusive, failover and
shared~\cite{hid-sp18-517-pulsar-apache}.  Pulsar can be installed and
run in standalone mode or standalone cluster, it can also be run
multiple clusters. Pulsar installation involves installing an instance
which can be installed across clusters when installed in multi-cluster
environment. In this setup clusters can be running within the data
center or can span across multiple data centers.  Pulsar also support
geo-replication so the clusters can replicate with each other. Pulsar
can also be installed on Kubernetes on Google Kubernetes or
AWS~\cite{hid-sp18-517-pulsar-apache}.
