\section{Sqoop}

The primary application of Sqoop is data transfer between the
traditional or relational database management systems and Hadoop
platforms. It also has the capability to transfer data from mainframes
to Hadoop. Sqoop works with Oracle, MySQL and can import data from
these sources into the Hadoop distributed File systems or HDFS. In
addition it can also transform data in map-reduce or even export it to
the database such as Oracle.  Sqoop works in batch mode and cannot
move data real time.  ~\cite{hid-sp18-517-Sqoop}. Sqoop relies on the
database to describe the schema of the data being imported. It uses
MapReduce to import and export the data, which provides parallel
operation as well as fault tolerance.  For databases, Sqoop reads the
table row-by-row into HDFS.  For mainframe datasets, Sqoop reads
records from each mainframe dataset into HDFS. The output of this
import process is a set of files containing a copy of the imported
table or datasets. Since the import process runs in parallel processes
each process creates a file causing multiple files being
created. These text files can use different delimiters such as comma,
pipe and so on ~\cite{hid-sp18-517-Sqoop}. Sqoop relies on the
database to describe the schema of the data being imported. It uses
MapReduce to import and export the data, which provides parallel
operation as well as fault tolerance.  For databases, Sqoop reads the
table row-by-row into HDFS.  For mainframe datasets, Sqoop reads
records from each mainframe dataset into HDFS. The output of this
import process is a set of files containing a copy of the imported
table or datasets. Since the import process runs in parallel processes
each process creates a file causing multiple files being
created. These text files can use different delimiters such as comma,
pipe and so on~\cite{hid-sp18-517-Sqoop}.
\footnote{citation wrongly placed}